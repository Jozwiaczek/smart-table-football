\chapter{Podsumowanie oraz wnioski}
\label{ch:funplenop}

Tematem niniejszej pracy była realizacja inteligentnego systemu do piłkarzyków stołowych.

Głównym celem pracy było omówienie zrealizowanego fizycznego prototypu bramki, systemu urozmaicającego fizyczną grę z wykorzystaniem stołu do piłkarzyków oraz przedstawienie wybranych rozwiązań, technologii i architektury.

Projekt został zrealizowany jako dwie aplikacje internetowe z podejściem SPA oraz PWA, główny serwer oraz mini komputer Raspberry Pi, do którego podłączona została kamera nagrywająca powtórki goli, czujniki bramek oraz diody sygnalizujące stan gry.

Wraz z realizacją projektu rozwiązane zostały problemy dużego repozytorium kodu i liczności różnych funkcjonalności z podejściem monolitycznego repozytorium oraz jednego środowiska języka programowania. Tym samym prototyp bramki przedstawia pełne założenia wykorzystania bramki razem ze zbudowany systemem, które mogą posłużyć przyszłej budowie realnego stołu do piłkarzyków. Ponadto zrealizowana aplikacja dzięki zastosowaniu podejściu PWA obsługuje pełen zakres urządzeń na różnych platformach z możliwością instalacji na ekranie głównym i wyglądem natywnej aplikacji.