% zmiana nazwy abstraktu

\begin{abstract}
Powstanie pierwszego stołu do piłkarzyków szacuje się na przełom XX wieku. Nie ma jednoznacznego źródła, z którego wynikała by data, miejsce oraz jego wynalazca. Jego popularność wzrosła po zakończeniu I wojny światowej, jako forma rekreacji oraz rehabilitacji. 
Stół do piłkarzyków uzyskał swój drugi największy wzrost popularności dzięki stworzeniu pierwszych zarobkowych automatów do gry w piłkarzyki. Pierwsze tego typu stoły pojawiły się w latach sześćdziesiątych. \cite{TableFootballHistory}

Współcześnie stoły do piłkarzyków stały się o wiele bardziej dostępne i tańsze w zakupie. Najczęściej możemy je znaleźć w pokojach rozgrywek, firmach, hotelach ale i również w prywatnych domach. 

Mimo relatywnie niskiej ceny, większości stołów dostępnych na rynku, żaden nie oferuje cyfrowego rozszerzenia w celu lepszych doświadczeń użytkowników, poprzez możliwość śledzenia rozgranych gier, rejestracji graczy oraz analizy otrzymanych wyników. Również nie jest możliwe dokupienie żadnych komponentów, które taką funkcjonalność by oferowały.

Pomysł na realizacje systemu modularnego do stołu piłkarzykowego powstał podczas jednej z rozgrywek w piłkarzyki w firmie, w której pracowałem. Podczas faktycznych rozgrywek brakowało mi i moim współpracownikom jednolitego i łatwego w dostępie systemu do śledzenia postępów w grze, zautomatyzowanego licznika goli, monitorowania oraz odtwarzania najciekawszych momentów.

Obserwując sytuację zarówno w mojej, jak i innych firmach, zauważyłem zdecydowany brak tego rodzaju funkcjonalności, dlatego też powstał projekt 'Smart Table Football'.
\end{abstract}


    