% zmiana nazwy abstraktu

\begin{abstract}
Powstanie pierwszego stołu do piłkarzyków datuje się na okres przełomu XX wieku . Rozbieżność ta bierze się z faktu, że nie ma jednoznacznego źródła z którego wynikała by data, miejsce oraz ich wynalazca. Ich popularność wzrosła po zakończeniu I wojny światowej, jako forma rekreacji oraz rehabilitacji. 
Swój drugi największy wzrost popularności uzyskały dzięki stworzeniu pierwszych zarobkowych automatów do gry w piłkarzyki. Pierwsze tego typu stoły pojawiły się w latach sześćdziesiątych. \cite{TableFootballHistory}

Współcześnie stoły do piłkarzyków stały się o wiele łatwiejsze i tańsze w zakupie. Najczęściej możemy je znaleźć w pokojach rozgrywek, firmach, hotelach ale i również prywatnych domach. 

Mimo relatywnie niskiej ceny większości stołów dostępnych na rynku, żaden nie oferuje cyfrowego rozszerzenia w celu lepszych doświadczeń użytkowników. Również nie jest możliwe dokupienie żadnych komponentów które taką funkcjonalność by oferowały.

Pomysł na realizacje systemu modularnego do stołu piłkarzykowego powstał podczas jednej z rozgrywek w piłkarzyki w firmie w której pracowałem. Podczas faktycznych rozgrywek brakowało mi i moim współpracownikom jednolitego i łatwego w dostępie systemu do śledzenia postępów w grze, zautomatyzowanego licznika goli, monitorowania oraz odtwarzania najciekawszych momentów.

Obserwując sytuację zarówno w mojej, jak i innych firmach zauważyłem zdecydowany brak tego rodzaju funkcjonalności, dlatego też powstał projekt "Smart Table Football".
\end{abstract}


    