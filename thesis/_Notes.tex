Problem -> Możliwośći -> Analiza -> Uzasadnienie wyboru
        
WSTĘP
- Rozwiązania przyjęte w pracy
- Rezultaty pracy
- Organizacja pracy
co opisuje w pierwszym etapie co w drugim itd
        
-SOFTWARE
-wstęp - co zostanie przedstawione w rozdziale i do czego będą odwołania
-ogół
    - Aplikacja Webowa
        - Czemu tylko webowka
        - rozwój
        - PWA
    -javascript
        -o języku (ECMA)
        -wersje języka
        -rozwój języka
        -ciekawe rzeczy które wykorzystałem
        -funkcyjnie
        -porównanie to ts
    -stack
        -co to stacku mern
        -czemu nie mean
    -Diagramy projektu
        - baza danych
        - komunikacja
    -Podejście mongoRepo vs singleRepo
    - Project core - schema
    -IDE - Webstorm oraz DevDependecies
        -Etheral
        -Linter
    -Git - github

-backend
    -Połączenie rest i socket
        -socket.io
    -Node
        - porównanie wersji i ich prędkości
        - Node vs inne środowiska
        - Deno.js
     -MongoDb
        -mongoose
        -porównanie z innymi bazami danych
    -Express - krótki opis
    -Feathers
        - opisać jak działa
    -Electron - krótko
    -GoogleDrive - integracja i możliwości
    -OnOff - GPIO
    -Sendgrid - krótko o mailingu
    -Robienie zdjęć i nagrywania - opisać dokładnie i czemu nie biblioteka
 
 
    

-fronted
    -HTML
    -CSS
    -react
        -react vs angular vs vue vs vanilajs vs Gatsby
        -virtual dom
        -hooki
        -o współbiezności
    -Material Design
    -Material ui
    -react admin
    -i18n i theming

-devops
    -heroku
    -CircleCI





DO HARDWARE
-Druk 3D
    -Fusion 360
    -Cura
    -Octoprint
    - Materiał PLA
        - że PLA
        - Dlaczego nie inne
-Raspberry pi 4
    -różnica z 3 wersją
    -porównanie z arduino
-picam
    -czemu picam
    
Jakość́ oprogramowania
według bytów
    
SUMARYCZNE DZIAŁANIE APLIKACJI
