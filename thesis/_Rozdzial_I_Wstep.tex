\chapter{Wstęp}
\label{ch:funplenop}

\section{Cel pracy}
Celem mojej pracy jest omówienie zrealizowanego fizycznego prototypu bramki, systemu dla piłkarzyków oraz przedstawienie wybranych rozwiązań, technologii i architektury.

\section{Przyjęte rozwiązania w pracy}
System został zaimplementowany z użyciem języka JavaScript w środowisku uruchomieniowm Node.js w architekturze repozytorium monolitycznego z podziałem systemu na 7 pakietów, w którym każdy jest odpowiedzialny za inną częśc systemu. Fizyczna część pracy została zbudowana przy pomocy mikro komputera Raspberry Pi oraz druku 3D.

\section{Rezultaty pracy}
Rezultatem mojej pracy jest fizyczny prototyp bramki wydrukowanej w technologii druku 3D połączoną z mikro komputerem Raspberry Pi łączącym się z głównym serwerem systemu oraz aplikacją gracza i administratora.

\section{Organizacja pracy}
Poniższa praca została podzielona na osiem rozdziałów.
Rozdział Architektura opisuje fragmenty pracy związne z architekturą systemu, środowiskiem pracy oraz ogólnie przyjętymi rozwiązaniami. 
Kolejny rozdział 'Backend' przedstawia technologie oraz rozwiązania programistyczne z obaszru logiki biznesowej w systemie, przyjęte konkretnie w pakietach 'api', 'table' oraz 'table-manger'.
Rozdział `Frontend` opisuje zaplecze projektowe bliżej zbliżone temu co widzi docelowy użytkownik aplikacji, skupiając się tym samym na technologiach wykorzystywanych w pakietach 'admin', 'player' oraz 'ui-components'.
Następny rozdział 'Hardware' skupia się na przedstawieniu zagadnień związanych z fizyczną częścią projektu oraz narzędziami jakie zostały wykorzystane podczas pracy.
Rozdział 'Publikacja' opisuje wykorzystane podejście oraz narzędzia publikacji aplikacji końcowej.
Najważniejszym etapem pracy jest rozdział 'Aplikacja', przedstawiającym faktyczne działanie aplikacji oraz sposób jej uruchomienia z podziałem na 7 zdefiniowanych pakietów.
W ostatnim rozdziale znajduje się podsumowanie, omawiające zrealizowany cel na tle występujących podczas realizacji trudności.
