\chapter{Wstęp}
\label{ch:wstęp}

\section{Cel pracy}
Celem pracy jest omówienie zrealizowanego fizycznego prototypu bramki, systemu wspomagającego fizyczną grę w piłkarzyki stołowe oraz przedstawienie wybranych rozwiązań, technologii i architektury.

\section{Przyjęte rozwiązania w pracy}
System został zaimplementowany w całości z użyciem języka JavaScript w środowisku uruchomieniowym Node w architekturze repozytorium monolitycznego z podziałem systemu na 7 pakietów, w którym każdy jest odpowiedzialny za inną część systemu. Fizyczna część pracy została wykonana z wykorzystaniem mikro komputera Raspberry Pi oraz druku 3D.

\section{Rezultaty pracy}
Rezultatem pracy jest fizyczny prototyp bramki wydrukowanej w technologii druku 3D połączony z mini komputerem Raspberry Pi łączącym się z głównym serwerem systemu oraz aplikacją gracza i administratora.

\section{Organizacja pracy}
Poniższa praca została podzielona na osiem rozdziałów. Rozdział pierwszy stanowi wstęp do pracy. \ref{ch:architektura} - Architektura opisuje fragmenty pracy związanej z architekturą systemu, środowiskiem pracy oraz ogólnie przyjętymi rozwiązaniami. 
Kolejny rozdział \ref{ch:backend} - Backend przedstawia technologie oraz rozwiązania programistyczne z obszaru logiki biznesowej w systemie, przyjęte konkretnie w pakietach 'api', 'table' oraz 'table-manager'.
Rozdział \ref{ch:frontend} - Frontend opisuje zaplecze projektowe zawierające aplikacje z interfejsami graficznymi, skupiając się tym samym na technologiach wykorzystywanych w pakietach 'admin', 'player' oraz 'ui-components'.
Następny rozdział \ref{ch:hardware} - Hardware skupia się na przedstawieniu zagadnień związanych z fizyczną częścią projektu oraz narzędziami jakie zostały wykorzystane podczas pracy.
Rozdział \ref{ch:publikacja} - Publikacja opisuje wykorzystane podejście oraz narzędzia publikacji aplikacji końcowej.
Najważniejszym etapem pracy jest rozdział \ref{ch:application} - Aplikacja, przedstawiający faktyczne działanie aplikacji oraz sposób jej uruchomienia z podziałem na 7 zdefiniowanych pakietów.
W ostatnim rozdziale \ref{ch:podsumowanie} - Podsumowanie, znajduje się podsumowanie, omawiające zrealizowany cel.
