\chapter{Backend}
\label{ch:funplenop}
W tym rozdziale zostaną przedstawione technologie oraz rozwiązania programistyczne z obaszru logiki biznesowej w systemie, przyjęte konkretnie w pakietach 'api', 'table' oraz 'table-manger'.

\section{Node.js}

Wykorzystywany język JavaScript bezpośrednio nie oferuje komunikacji między systemem a aplikacją, jego powszechne środowisko uruchmieniowe to przeglądarka. Założeniami natomiast omawianej pracy było międzyinnymi stworzenie serwerów. W celu realizacji takiej funkcjonalności w systemie zostało zastosowane wieloplatformowe środowisko uruchomieniowe jakim jest node.js. Pozwala ono na tworzenie aplikacji w języku JavaScript, zazwyczaj są to aplikacje serwerowe, które umożliwiają swoje działanie poza przeglądarką. Tym samym Node.js udostępnia interfejs komunikacyjny z systemem, który pozwala na np. odczyt katalogów czy  zapis plików. Silnik omawianego środowiska uruchomieniowego oparty jest na silniku V8, rozwijanym przez firmę google, czyli tym samym co przeglądakra Chrome. Node.js jest bardzo dynamicznie rozwiającą się platformą, systematycznie co roku publikowana jest nowa wersja(https://nodejs.org/en/about/releases/). Omawiany projekt systemu zakłada użycie przynajmniej 12 wersji node.js (w chwili pisania pracy oficalnie aktualna wersja to 14). Poza wieloma wyżej wymienionymi zaletami systemu, wokół tej tenchologi znajduje się ogromna społeczność, która rozwija wiele dodatków oraz oferuje swoje wsparcie w interncie. Wedle ankiety z 2020 roku serwisu StackOverflow, NodeJS uzyskał pierwsze miejsce w kategorii wykorzystywanych technologii (dokładnie 51.9 procent). 
% https://insights.stackoverflow.com/survey/2020#technology-other-frameworks-libraries-and-tools-professional-developers3
% https://developer.mozilla.org/pl/docs/Learn/Server-side/Express_Nodejs/Introduction

\section{Feathers.js}
% https://feathersjs.com/
Zbudowane w systemie serwery opierają się o bibliotkę Express.js, która służy do obslugi zapytań HTTP, konfiguracji seerwerów. Całość u podstawy oparta jest własnie o powyżej opisywany Node.js. Ze względu na swoją minimalistyczność Express.js posiada wiele zintegrowanych biliotek/nakładek, które automatyzują wiele powtarzającyh się operacji oraz udostępniają zestaw narzedzi ułatwiających pracę w budowie popularnych funkcjonalności. Użytą w systemie nakładą na Express.js jest kolejna biblioteka Feathers.js. Narzedzie te zorientowane jest na budowanie aplikacji serwerowych z wykorzystaniem serwisów oraz hooków. Głównym założeniem jest zautomatyzowanie budowania aplikacji serwerowych, pozostawiajać przy tym pełną kontrolę oraz zachęcając/zapewniając dobre praktyki oprogramowania. Serwisy są obiektami reprezentujacymi predefiniowane zestawy metod CRUD (Stwórz, Wylistuj, Zzaktualizuj, Usuń). Hooki zaś są wartwą pośrednią serwisów. Umożlwiają one wpięcie się w cykl życia pojedynczych metod serwisów (przed, po i na błąd podczas ich wykonwania) oraz na wykonwyanie zdefiniowanych ciągów metod w tych wybranych etapach. 

Podczas budowy pakietu 'api' został wykorzystany również mechanizm tej bilioteki do autogenerowania początkowej z  konfiguracji aplikacji oraz generowania serwisów i hooków na podstawie zbudowanej struktury.

Kolejną bardzo istotną funkcjonalnością feathers.js zdefiniowaną w systemie jest adapter bazy danych 'feathers-mongose' oraz '@feathersjs/authentication', mechanizm autentykacji. Adapter bazy danych pozwala na prostą konfiguracje aplikacji działającej z bazą danych mongodb oraz deklaracje wykorzystywanych modeli opisujących strukture danych. Natomiast mechanizm autentykacji to dodatek oferujący narzedzia służące tworzeniu mechanizmu autentykacji. W systemie został użyty do utworzenia autentykacji ze startegią JWT (JavaScript Web Token) z algorytmem haszowania 'HS256'. 

Łącząc te wszystkie funkcjonalności Feathers.js końcowo zwraca zestaw zabezpieczonych mechanizmem autentykacji endpointów w architekturze REST (Representail State Transfer), umożliwiając w ten sposób pozostałym aplikacją na operacje serwerowe i bazodanowe.

\section{MongoDB}
mongoose (modelowanie danych oraz operacje -- middleware między serwerem a mongodb)
mongodb
połączenie z feathers
'code as a infrustracure'

\section{Socket.io}

\section{Docker}
problem z uruchaminiem bazy danych w nowych środowiskach
Opis
Ze uruchamaia bazę danych lokalnie

\section{GoogleDrive}
- Zapisywanie zdjęć
- API

\section{OnOff - obsługa GPIO}
- OPIS
- Prosta obsługa GPIO

\section{System mailngowy}
- Opis
- Etheral
- SendGrid w rozdziale publikacja
- Wysyłka maili w połączeniu z nodemailer