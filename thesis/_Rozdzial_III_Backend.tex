\chapter{Backend}
\label{ch:funplenop}
W tym rozdziale zostaną przedstawione technologie oraz rozwiązania programistyczne z obaszru logiki biznesowej w systemie, przyjęte konkretnie w pakietach 'api', 'table' oraz 'table-manger'.

\section{Node.js}

% https://developer.mozilla.org/pl/docs/Learn/Server-side/Express_Nodejs/Introduction
wieloplatformowe środowisko uruchomieniowe do tworzenia aplikacji w języku JavaScript (działa poza przeglądarką) do tworzenia aplikacji backendowych (praca z serwerami)
komunikacja oraz praca z systemem operacyjnym
wykorzystuje zoptymalizowany silnik V8 od firmy Google
jednowątkowy ale bardzo zoptymalizowany
wykres szybkości na przestrzeni wersji
wbudowany w node npm (node package manager)
bardzo duża społeczność


\section{Feathers.js}
% https://feathersjs.com/
Nakładka na express
Zorientowany na serwisy i hooki
Generowanie serwisów
Generwoanie hooków (warstwa pośrednia dla wszystkich operacji serwisów z podziałem na 'before', 'after', 'errors', łączenie hooków w ciągi)
Adaptery bazy danych (feathers-mongoose)
Autentykacja oparta na strategii JWT poprzez zapytania http


\section{REST i Sockets}
Opis Rest
Opis Socket.IO
Co realizują

\section{MongoDB}
mongoose
mongodb
połączenie z feathers
'code as a infrustracure'

\section{Docker}
problem z uruchaminiem bazy danych w nowych środowiskach
Opis
Ze uruchamaia bazę danych lokalnie

\section{GoogleDrive}
- Zapisywanie zdjęć
- API

\section{OnOff - obsługa GPIO}
- OPIS
- Prosta obsługa GPIO

\section{System mailngowy}
- Opis
- Etheral
- SendGrid w rozdziale publikacja
- Wysyłka maili w połączeniu z nodemailer